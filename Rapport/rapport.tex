\documentclass[a4paper]{article}
\usepackage[T1]{fontenc}
\usepackage[utf8]{inputenc}
\usepackage{lmodern}
\usepackage{amsmath,amssymb}
\usepackage[top=3cm,bottom=2cm,left=2cm,right=2cm]{geometry}
\usepackage{fancyhdr}
\usepackage{mathabx}
\usepackage{esvect,esint}
\usepackage{xcolor}
\usepackage{tikz,circuitikz}\usetikzlibrary{calc}
\usepackage{graphicx}
\title{Systèmes d'exploitation\\Rapport de projet - Réseaux de Kahn}
\author{Sylvain Brisson - Arthur Rousseau}
\date{}
\parskip1em\parindent1pt\let\ds\displaystyle

\begin{document}
    \maketitle
    \noindent
    \section{Implémentation de réseaux de Kahn}
    \subsection{Version Unix}
    \subsection{Version sur le réseau}
    \subsection{Version séquentielle}
        On utilise une queue pour stocker les actions à effectuer (notamment pour le \texttt{doco}), avec des fonctions \texttt{add} pour ajouter une action à la queue et \texttt{suivant} pour exécuter la première action de la queue.\\
        La création d'un processus se ramène alors à \texttt{add} et l'exécution à \texttt{suivant}.
    \subsection{Version MPI}
    \section{Applications}
    \subsection{Crible}
        Le crible présenté est inspiré de celui du TP5. Il fait passer chaque entier par une suite de filtres (\texttt{sift}), où chaque filtre teste la divisibilité par un nombre premier. Si on rencontre un nombre premier, on crée un nouveau filtre.
    \subsection{Algorithme de Strassen}

\end{document}